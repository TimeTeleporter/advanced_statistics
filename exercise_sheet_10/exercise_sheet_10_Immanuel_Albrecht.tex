\documentclass[11pt]{article}

    \usepackage[breakable]{tcolorbox}
    \usepackage{parskip} % Stop auto-indenting (to mimic markdown behaviour)
    
    \usepackage{iftex}
    \ifPDFTeX
    	\usepackage[T1]{fontenc}
    	\usepackage{mathpazo}
    \else
    	\usepackage{fontspec}
    \fi

    % Basic figure setup, for now with no caption control since it's done
    % automatically by Pandoc (which extracts ![](path) syntax from Markdown).
    \usepackage{graphicx}
    % Maintain compatibility with old templates. Remove in nbconvert 6.0
    \let\Oldincludegraphics\includegraphics
    % Ensure that by default, figures have no caption (until we provide a
    % proper Figure object with a Caption API and a way to capture that
    % in the conversion process - todo).
    \usepackage{caption}
    \DeclareCaptionFormat{nocaption}{}
    \captionsetup{format=nocaption,aboveskip=0pt,belowskip=0pt}

    \usepackage{float}
    \floatplacement{figure}{H} % forces figures to be placed at the correct location
    \usepackage{xcolor} % Allow colors to be defined
    \usepackage{enumerate} % Needed for markdown enumerations to work
    \usepackage{geometry} % Used to adjust the document margins
    \usepackage{amsmath} % Equations
    \usepackage{amssymb} % Equations
    \usepackage{textcomp} % defines textquotesingle
    % Hack from http://tex.stackexchange.com/a/47451/13684:
    \AtBeginDocument{%
        \def\PYZsq{\textquotesingle}% Upright quotes in Pygmentized code
    }
    \usepackage{upquote} % Upright quotes for verbatim code
    \usepackage{eurosym} % defines \euro
    \usepackage[mathletters]{ucs} % Extended unicode (utf-8) support
    \usepackage{fancyvrb} % verbatim replacement that allows latex
    \usepackage{grffile} % extends the file name processing of package graphics 
                         % to support a larger range
    \makeatletter % fix for old versions of grffile with XeLaTeX
    \@ifpackagelater{grffile}{2019/11/01}
    {
      % Do nothing on new versions
    }
    {
      \def\Gread@@xetex#1{%
        \IfFileExists{"\Gin@base".bb}%
        {\Gread@eps{\Gin@base.bb}}%
        {\Gread@@xetex@aux#1}%
      }
    }
    \makeatother
    \usepackage[Export]{adjustbox} % Used to constrain images to a maximum size
    \adjustboxset{max size={0.9\linewidth}{0.9\paperheight}}

    % The hyperref package gives us a pdf with properly built
    % internal navigation ('pdf bookmarks' for the table of contents,
    % internal cross-reference links, web links for URLs, etc.)
    \usepackage{hyperref}
    % The default LaTeX title has an obnoxious amount of whitespace. By default,
    % titling removes some of it. It also provides customization options.
    \usepackage{titling}
    \usepackage{longtable} % longtable support required by pandoc >1.10
    \usepackage{booktabs}  % table support for pandoc > 1.12.2
    \usepackage[inline]{enumitem} % IRkernel/repr support (it uses the enumerate* environment)
    \usepackage[normalem]{ulem} % ulem is needed to support strikethroughs (\sout)
                                % normalem makes italics be italics, not underlines
    \usepackage{mathrsfs}
    

    
    % Colors for the hyperref package
    \definecolor{urlcolor}{rgb}{0,.145,.698}
    \definecolor{linkcolor}{rgb}{.71,0.21,0.01}
    \definecolor{citecolor}{rgb}{.12,.54,.11}

    % ANSI colors
    \definecolor{ansi-black}{HTML}{3E424D}
    \definecolor{ansi-black-intense}{HTML}{282C36}
    \definecolor{ansi-red}{HTML}{E75C58}
    \definecolor{ansi-red-intense}{HTML}{B22B31}
    \definecolor{ansi-green}{HTML}{00A250}
    \definecolor{ansi-green-intense}{HTML}{007427}
    \definecolor{ansi-yellow}{HTML}{DDB62B}
    \definecolor{ansi-yellow-intense}{HTML}{B27D12}
    \definecolor{ansi-blue}{HTML}{208FFB}
    \definecolor{ansi-blue-intense}{HTML}{0065CA}
    \definecolor{ansi-magenta}{HTML}{D160C4}
    \definecolor{ansi-magenta-intense}{HTML}{A03196}
    \definecolor{ansi-cyan}{HTML}{60C6C8}
    \definecolor{ansi-cyan-intense}{HTML}{258F8F}
    \definecolor{ansi-white}{HTML}{C5C1B4}
    \definecolor{ansi-white-intense}{HTML}{A1A6B2}
    \definecolor{ansi-default-inverse-fg}{HTML}{FFFFFF}
    \definecolor{ansi-default-inverse-bg}{HTML}{000000}

    % common color for the border for error outputs.
    \definecolor{outerrorbackground}{HTML}{FFDFDF}

    % commands and environments needed by pandoc snippets
    % extracted from the output of `pandoc -s`
    \providecommand{\tightlist}{%
      \setlength{\itemsep}{0pt}\setlength{\parskip}{0pt}}
    \DefineVerbatimEnvironment{Highlighting}{Verbatim}{commandchars=\\\{\}}
    % Add ',fontsize=\small' for more characters per line
    \newenvironment{Shaded}{}{}
    \newcommand{\KeywordTok}[1]{\textcolor[rgb]{0.00,0.44,0.13}{\textbf{{#1}}}}
    \newcommand{\DataTypeTok}[1]{\textcolor[rgb]{0.56,0.13,0.00}{{#1}}}
    \newcommand{\DecValTok}[1]{\textcolor[rgb]{0.25,0.63,0.44}{{#1}}}
    \newcommand{\BaseNTok}[1]{\textcolor[rgb]{0.25,0.63,0.44}{{#1}}}
    \newcommand{\FloatTok}[1]{\textcolor[rgb]{0.25,0.63,0.44}{{#1}}}
    \newcommand{\CharTok}[1]{\textcolor[rgb]{0.25,0.44,0.63}{{#1}}}
    \newcommand{\StringTok}[1]{\textcolor[rgb]{0.25,0.44,0.63}{{#1}}}
    \newcommand{\CommentTok}[1]{\textcolor[rgb]{0.38,0.63,0.69}{\textit{{#1}}}}
    \newcommand{\OtherTok}[1]{\textcolor[rgb]{0.00,0.44,0.13}{{#1}}}
    \newcommand{\AlertTok}[1]{\textcolor[rgb]{1.00,0.00,0.00}{\textbf{{#1}}}}
    \newcommand{\FunctionTok}[1]{\textcolor[rgb]{0.02,0.16,0.49}{{#1}}}
    \newcommand{\RegionMarkerTok}[1]{{#1}}
    \newcommand{\ErrorTok}[1]{\textcolor[rgb]{1.00,0.00,0.00}{\textbf{{#1}}}}
    \newcommand{\NormalTok}[1]{{#1}}
    
    % Additional commands for more recent versions of Pandoc
    \newcommand{\ConstantTok}[1]{\textcolor[rgb]{0.53,0.00,0.00}{{#1}}}
    \newcommand{\SpecialCharTok}[1]{\textcolor[rgb]{0.25,0.44,0.63}{{#1}}}
    \newcommand{\VerbatimStringTok}[1]{\textcolor[rgb]{0.25,0.44,0.63}{{#1}}}
    \newcommand{\SpecialStringTok}[1]{\textcolor[rgb]{0.73,0.40,0.53}{{#1}}}
    \newcommand{\ImportTok}[1]{{#1}}
    \newcommand{\DocumentationTok}[1]{\textcolor[rgb]{0.73,0.13,0.13}{\textit{{#1}}}}
    \newcommand{\AnnotationTok}[1]{\textcolor[rgb]{0.38,0.63,0.69}{\textbf{\textit{{#1}}}}}
    \newcommand{\CommentVarTok}[1]{\textcolor[rgb]{0.38,0.63,0.69}{\textbf{\textit{{#1}}}}}
    \newcommand{\VariableTok}[1]{\textcolor[rgb]{0.10,0.09,0.49}{{#1}}}
    \newcommand{\ControlFlowTok}[1]{\textcolor[rgb]{0.00,0.44,0.13}{\textbf{{#1}}}}
    \newcommand{\OperatorTok}[1]{\textcolor[rgb]{0.40,0.40,0.40}{{#1}}}
    \newcommand{\BuiltInTok}[1]{{#1}}
    \newcommand{\ExtensionTok}[1]{{#1}}
    \newcommand{\PreprocessorTok}[1]{\textcolor[rgb]{0.74,0.48,0.00}{{#1}}}
    \newcommand{\AttributeTok}[1]{\textcolor[rgb]{0.49,0.56,0.16}{{#1}}}
    \newcommand{\InformationTok}[1]{\textcolor[rgb]{0.38,0.63,0.69}{\textbf{\textit{{#1}}}}}
    \newcommand{\WarningTok}[1]{\textcolor[rgb]{0.38,0.63,0.69}{\textbf{\textit{{#1}}}}}
    
    
    % Define a nice break command that doesn't care if a line doesn't already
    % exist.
    \def\br{\hspace*{\fill} \\* }
    % Math Jax compatibility definitions
    \def\gt{>}
    \def\lt{<}
    \let\Oldtex\TeX
    \let\Oldlatex\LaTeX
    \renewcommand{\TeX}{\textrm{\Oldtex}}
    \renewcommand{\LaTeX}{\textrm{\Oldlatex}}
    % Document parameters
    % Document title
    \title{exercise\_sheet\_10\_Immanuel\_Albrecht}
    
    
    
    
    
% Pygments definitions
\makeatletter
\def\PY@reset{\let\PY@it=\relax \let\PY@bf=\relax%
    \let\PY@ul=\relax \let\PY@tc=\relax%
    \let\PY@bc=\relax \let\PY@ff=\relax}
\def\PY@tok#1{\csname PY@tok@#1\endcsname}
\def\PY@toks#1+{\ifx\relax#1\empty\else%
    \PY@tok{#1}\expandafter\PY@toks\fi}
\def\PY@do#1{\PY@bc{\PY@tc{\PY@ul{%
    \PY@it{\PY@bf{\PY@ff{#1}}}}}}}
\def\PY#1#2{\PY@reset\PY@toks#1+\relax+\PY@do{#2}}

\expandafter\def\csname PY@tok@w\endcsname{\def\PY@tc##1{\textcolor[rgb]{0.73,0.73,0.73}{##1}}}
\expandafter\def\csname PY@tok@c\endcsname{\let\PY@it=\textit\def\PY@tc##1{\textcolor[rgb]{0.25,0.50,0.50}{##1}}}
\expandafter\def\csname PY@tok@cp\endcsname{\def\PY@tc##1{\textcolor[rgb]{0.74,0.48,0.00}{##1}}}
\expandafter\def\csname PY@tok@k\endcsname{\let\PY@bf=\textbf\def\PY@tc##1{\textcolor[rgb]{0.00,0.50,0.00}{##1}}}
\expandafter\def\csname PY@tok@kp\endcsname{\def\PY@tc##1{\textcolor[rgb]{0.00,0.50,0.00}{##1}}}
\expandafter\def\csname PY@tok@kt\endcsname{\def\PY@tc##1{\textcolor[rgb]{0.69,0.00,0.25}{##1}}}
\expandafter\def\csname PY@tok@o\endcsname{\def\PY@tc##1{\textcolor[rgb]{0.40,0.40,0.40}{##1}}}
\expandafter\def\csname PY@tok@ow\endcsname{\let\PY@bf=\textbf\def\PY@tc##1{\textcolor[rgb]{0.67,0.13,1.00}{##1}}}
\expandafter\def\csname PY@tok@nb\endcsname{\def\PY@tc##1{\textcolor[rgb]{0.00,0.50,0.00}{##1}}}
\expandafter\def\csname PY@tok@nf\endcsname{\def\PY@tc##1{\textcolor[rgb]{0.00,0.00,1.00}{##1}}}
\expandafter\def\csname PY@tok@nc\endcsname{\let\PY@bf=\textbf\def\PY@tc##1{\textcolor[rgb]{0.00,0.00,1.00}{##1}}}
\expandafter\def\csname PY@tok@nn\endcsname{\let\PY@bf=\textbf\def\PY@tc##1{\textcolor[rgb]{0.00,0.00,1.00}{##1}}}
\expandafter\def\csname PY@tok@ne\endcsname{\let\PY@bf=\textbf\def\PY@tc##1{\textcolor[rgb]{0.82,0.25,0.23}{##1}}}
\expandafter\def\csname PY@tok@nv\endcsname{\def\PY@tc##1{\textcolor[rgb]{0.10,0.09,0.49}{##1}}}
\expandafter\def\csname PY@tok@no\endcsname{\def\PY@tc##1{\textcolor[rgb]{0.53,0.00,0.00}{##1}}}
\expandafter\def\csname PY@tok@nl\endcsname{\def\PY@tc##1{\textcolor[rgb]{0.63,0.63,0.00}{##1}}}
\expandafter\def\csname PY@tok@ni\endcsname{\let\PY@bf=\textbf\def\PY@tc##1{\textcolor[rgb]{0.60,0.60,0.60}{##1}}}
\expandafter\def\csname PY@tok@na\endcsname{\def\PY@tc##1{\textcolor[rgb]{0.49,0.56,0.16}{##1}}}
\expandafter\def\csname PY@tok@nt\endcsname{\let\PY@bf=\textbf\def\PY@tc##1{\textcolor[rgb]{0.00,0.50,0.00}{##1}}}
\expandafter\def\csname PY@tok@nd\endcsname{\def\PY@tc##1{\textcolor[rgb]{0.67,0.13,1.00}{##1}}}
\expandafter\def\csname PY@tok@s\endcsname{\def\PY@tc##1{\textcolor[rgb]{0.73,0.13,0.13}{##1}}}
\expandafter\def\csname PY@tok@sd\endcsname{\let\PY@it=\textit\def\PY@tc##1{\textcolor[rgb]{0.73,0.13,0.13}{##1}}}
\expandafter\def\csname PY@tok@si\endcsname{\let\PY@bf=\textbf\def\PY@tc##1{\textcolor[rgb]{0.73,0.40,0.53}{##1}}}
\expandafter\def\csname PY@tok@se\endcsname{\let\PY@bf=\textbf\def\PY@tc##1{\textcolor[rgb]{0.73,0.40,0.13}{##1}}}
\expandafter\def\csname PY@tok@sr\endcsname{\def\PY@tc##1{\textcolor[rgb]{0.73,0.40,0.53}{##1}}}
\expandafter\def\csname PY@tok@ss\endcsname{\def\PY@tc##1{\textcolor[rgb]{0.10,0.09,0.49}{##1}}}
\expandafter\def\csname PY@tok@sx\endcsname{\def\PY@tc##1{\textcolor[rgb]{0.00,0.50,0.00}{##1}}}
\expandafter\def\csname PY@tok@m\endcsname{\def\PY@tc##1{\textcolor[rgb]{0.40,0.40,0.40}{##1}}}
\expandafter\def\csname PY@tok@gh\endcsname{\let\PY@bf=\textbf\def\PY@tc##1{\textcolor[rgb]{0.00,0.00,0.50}{##1}}}
\expandafter\def\csname PY@tok@gu\endcsname{\let\PY@bf=\textbf\def\PY@tc##1{\textcolor[rgb]{0.50,0.00,0.50}{##1}}}
\expandafter\def\csname PY@tok@gd\endcsname{\def\PY@tc##1{\textcolor[rgb]{0.63,0.00,0.00}{##1}}}
\expandafter\def\csname PY@tok@gi\endcsname{\def\PY@tc##1{\textcolor[rgb]{0.00,0.63,0.00}{##1}}}
\expandafter\def\csname PY@tok@gr\endcsname{\def\PY@tc##1{\textcolor[rgb]{1.00,0.00,0.00}{##1}}}
\expandafter\def\csname PY@tok@ge\endcsname{\let\PY@it=\textit}
\expandafter\def\csname PY@tok@gs\endcsname{\let\PY@bf=\textbf}
\expandafter\def\csname PY@tok@gp\endcsname{\let\PY@bf=\textbf\def\PY@tc##1{\textcolor[rgb]{0.00,0.00,0.50}{##1}}}
\expandafter\def\csname PY@tok@go\endcsname{\def\PY@tc##1{\textcolor[rgb]{0.53,0.53,0.53}{##1}}}
\expandafter\def\csname PY@tok@gt\endcsname{\def\PY@tc##1{\textcolor[rgb]{0.00,0.27,0.87}{##1}}}
\expandafter\def\csname PY@tok@err\endcsname{\def\PY@bc##1{\setlength{\fboxsep}{0pt}\fcolorbox[rgb]{1.00,0.00,0.00}{1,1,1}{\strut ##1}}}
\expandafter\def\csname PY@tok@kc\endcsname{\let\PY@bf=\textbf\def\PY@tc##1{\textcolor[rgb]{0.00,0.50,0.00}{##1}}}
\expandafter\def\csname PY@tok@kd\endcsname{\let\PY@bf=\textbf\def\PY@tc##1{\textcolor[rgb]{0.00,0.50,0.00}{##1}}}
\expandafter\def\csname PY@tok@kn\endcsname{\let\PY@bf=\textbf\def\PY@tc##1{\textcolor[rgb]{0.00,0.50,0.00}{##1}}}
\expandafter\def\csname PY@tok@kr\endcsname{\let\PY@bf=\textbf\def\PY@tc##1{\textcolor[rgb]{0.00,0.50,0.00}{##1}}}
\expandafter\def\csname PY@tok@bp\endcsname{\def\PY@tc##1{\textcolor[rgb]{0.00,0.50,0.00}{##1}}}
\expandafter\def\csname PY@tok@fm\endcsname{\def\PY@tc##1{\textcolor[rgb]{0.00,0.00,1.00}{##1}}}
\expandafter\def\csname PY@tok@vc\endcsname{\def\PY@tc##1{\textcolor[rgb]{0.10,0.09,0.49}{##1}}}
\expandafter\def\csname PY@tok@vg\endcsname{\def\PY@tc##1{\textcolor[rgb]{0.10,0.09,0.49}{##1}}}
\expandafter\def\csname PY@tok@vi\endcsname{\def\PY@tc##1{\textcolor[rgb]{0.10,0.09,0.49}{##1}}}
\expandafter\def\csname PY@tok@vm\endcsname{\def\PY@tc##1{\textcolor[rgb]{0.10,0.09,0.49}{##1}}}
\expandafter\def\csname PY@tok@sa\endcsname{\def\PY@tc##1{\textcolor[rgb]{0.73,0.13,0.13}{##1}}}
\expandafter\def\csname PY@tok@sb\endcsname{\def\PY@tc##1{\textcolor[rgb]{0.73,0.13,0.13}{##1}}}
\expandafter\def\csname PY@tok@sc\endcsname{\def\PY@tc##1{\textcolor[rgb]{0.73,0.13,0.13}{##1}}}
\expandafter\def\csname PY@tok@dl\endcsname{\def\PY@tc##1{\textcolor[rgb]{0.73,0.13,0.13}{##1}}}
\expandafter\def\csname PY@tok@s2\endcsname{\def\PY@tc##1{\textcolor[rgb]{0.73,0.13,0.13}{##1}}}
\expandafter\def\csname PY@tok@sh\endcsname{\def\PY@tc##1{\textcolor[rgb]{0.73,0.13,0.13}{##1}}}
\expandafter\def\csname PY@tok@s1\endcsname{\def\PY@tc##1{\textcolor[rgb]{0.73,0.13,0.13}{##1}}}
\expandafter\def\csname PY@tok@mb\endcsname{\def\PY@tc##1{\textcolor[rgb]{0.40,0.40,0.40}{##1}}}
\expandafter\def\csname PY@tok@mf\endcsname{\def\PY@tc##1{\textcolor[rgb]{0.40,0.40,0.40}{##1}}}
\expandafter\def\csname PY@tok@mh\endcsname{\def\PY@tc##1{\textcolor[rgb]{0.40,0.40,0.40}{##1}}}
\expandafter\def\csname PY@tok@mi\endcsname{\def\PY@tc##1{\textcolor[rgb]{0.40,0.40,0.40}{##1}}}
\expandafter\def\csname PY@tok@il\endcsname{\def\PY@tc##1{\textcolor[rgb]{0.40,0.40,0.40}{##1}}}
\expandafter\def\csname PY@tok@mo\endcsname{\def\PY@tc##1{\textcolor[rgb]{0.40,0.40,0.40}{##1}}}
\expandafter\def\csname PY@tok@ch\endcsname{\let\PY@it=\textit\def\PY@tc##1{\textcolor[rgb]{0.25,0.50,0.50}{##1}}}
\expandafter\def\csname PY@tok@cm\endcsname{\let\PY@it=\textit\def\PY@tc##1{\textcolor[rgb]{0.25,0.50,0.50}{##1}}}
\expandafter\def\csname PY@tok@cpf\endcsname{\let\PY@it=\textit\def\PY@tc##1{\textcolor[rgb]{0.25,0.50,0.50}{##1}}}
\expandafter\def\csname PY@tok@c1\endcsname{\let\PY@it=\textit\def\PY@tc##1{\textcolor[rgb]{0.25,0.50,0.50}{##1}}}
\expandafter\def\csname PY@tok@cs\endcsname{\let\PY@it=\textit\def\PY@tc##1{\textcolor[rgb]{0.25,0.50,0.50}{##1}}}

\def\PYZbs{\char`\\}
\def\PYZus{\char`\_}
\def\PYZob{\char`\{}
\def\PYZcb{\char`\}}
\def\PYZca{\char`\^}
\def\PYZam{\char`\&}
\def\PYZlt{\char`\<}
\def\PYZgt{\char`\>}
\def\PYZsh{\char`\#}
\def\PYZpc{\char`\%}
\def\PYZdl{\char`\$}
\def\PYZhy{\char`\-}
\def\PYZsq{\char`\'}
\def\PYZdq{\char`\"}
\def\PYZti{\char`\~}
% for compatibility with earlier versions
\def\PYZat{@}
\def\PYZlb{[}
\def\PYZrb{]}
\makeatother


    % For linebreaks inside Verbatim environment from package fancyvrb. 
    \makeatletter
        \newbox\Wrappedcontinuationbox 
        \newbox\Wrappedvisiblespacebox 
        \newcommand*\Wrappedvisiblespace {\textcolor{red}{\textvisiblespace}} 
        \newcommand*\Wrappedcontinuationsymbol {\textcolor{red}{\llap{\tiny$\m@th\hookrightarrow$}}} 
        \newcommand*\Wrappedcontinuationindent {3ex } 
        \newcommand*\Wrappedafterbreak {\kern\Wrappedcontinuationindent\copy\Wrappedcontinuationbox} 
        % Take advantage of the already applied Pygments mark-up to insert 
        % potential linebreaks for TeX processing. 
        %        {, <, #, %, $, ' and ": go to next line. 
        %        _, }, ^, &, >, - and ~: stay at end of broken line. 
        % Use of \textquotesingle for straight quote. 
        \newcommand*\Wrappedbreaksatspecials {% 
            \def\PYGZus{\discretionary{\char`\_}{\Wrappedafterbreak}{\char`\_}}% 
            \def\PYGZob{\discretionary{}{\Wrappedafterbreak\char`\{}{\char`\{}}% 
            \def\PYGZcb{\discretionary{\char`\}}{\Wrappedafterbreak}{\char`\}}}% 
            \def\PYGZca{\discretionary{\char`\^}{\Wrappedafterbreak}{\char`\^}}% 
            \def\PYGZam{\discretionary{\char`\&}{\Wrappedafterbreak}{\char`\&}}% 
            \def\PYGZlt{\discretionary{}{\Wrappedafterbreak\char`\<}{\char`\<}}% 
            \def\PYGZgt{\discretionary{\char`\>}{\Wrappedafterbreak}{\char`\>}}% 
            \def\PYGZsh{\discretionary{}{\Wrappedafterbreak\char`\#}{\char`\#}}% 
            \def\PYGZpc{\discretionary{}{\Wrappedafterbreak\char`\%}{\char`\%}}% 
            \def\PYGZdl{\discretionary{}{\Wrappedafterbreak\char`\$}{\char`\$}}% 
            \def\PYGZhy{\discretionary{\char`\-}{\Wrappedafterbreak}{\char`\-}}% 
            \def\PYGZsq{\discretionary{}{\Wrappedafterbreak\textquotesingle}{\textquotesingle}}% 
            \def\PYGZdq{\discretionary{}{\Wrappedafterbreak\char`\"}{\char`\"}}% 
            \def\PYGZti{\discretionary{\char`\~}{\Wrappedafterbreak}{\char`\~}}% 
        } 
        % Some characters . , ; ? ! / are not pygmentized. 
        % This macro makes them "active" and they will insert potential linebreaks 
        \newcommand*\Wrappedbreaksatpunct {% 
            \lccode`\~`\.\lowercase{\def~}{\discretionary{\hbox{\char`\.}}{\Wrappedafterbreak}{\hbox{\char`\.}}}% 
            \lccode`\~`\,\lowercase{\def~}{\discretionary{\hbox{\char`\,}}{\Wrappedafterbreak}{\hbox{\char`\,}}}% 
            \lccode`\~`\;\lowercase{\def~}{\discretionary{\hbox{\char`\;}}{\Wrappedafterbreak}{\hbox{\char`\;}}}% 
            \lccode`\~`\:\lowercase{\def~}{\discretionary{\hbox{\char`\:}}{\Wrappedafterbreak}{\hbox{\char`\:}}}% 
            \lccode`\~`\?\lowercase{\def~}{\discretionary{\hbox{\char`\?}}{\Wrappedafterbreak}{\hbox{\char`\?}}}% 
            \lccode`\~`\!\lowercase{\def~}{\discretionary{\hbox{\char`\!}}{\Wrappedafterbreak}{\hbox{\char`\!}}}% 
            \lccode`\~`\/\lowercase{\def~}{\discretionary{\hbox{\char`\/}}{\Wrappedafterbreak}{\hbox{\char`\/}}}% 
            \catcode`\.\active
            \catcode`\,\active 
            \catcode`\;\active
            \catcode`\:\active
            \catcode`\?\active
            \catcode`\!\active
            \catcode`\/\active 
            \lccode`\~`\~ 	
        }
    \makeatother

    \let\OriginalVerbatim=\Verbatim
    \makeatletter
    \renewcommand{\Verbatim}[1][1]{%
        %\parskip\z@skip
        \sbox\Wrappedcontinuationbox {\Wrappedcontinuationsymbol}%
        \sbox\Wrappedvisiblespacebox {\FV@SetupFont\Wrappedvisiblespace}%
        \def\FancyVerbFormatLine ##1{\hsize\linewidth
            \vtop{\raggedright\hyphenpenalty\z@\exhyphenpenalty\z@
                \doublehyphendemerits\z@\finalhyphendemerits\z@
                \strut ##1\strut}%
        }%
        % If the linebreak is at a space, the latter will be displayed as visible
        % space at end of first line, and a continuation symbol starts next line.
        % Stretch/shrink are however usually zero for typewriter font.
        \def\FV@Space {%
            \nobreak\hskip\z@ plus\fontdimen3\font minus\fontdimen4\font
            \discretionary{\copy\Wrappedvisiblespacebox}{\Wrappedafterbreak}
            {\kern\fontdimen2\font}%
        }%
        
        % Allow breaks at special characters using \PYG... macros.
        \Wrappedbreaksatspecials
        % Breaks at punctuation characters . , ; ? ! and / need catcode=\active 	
        \OriginalVerbatim[#1,codes*=\Wrappedbreaksatpunct]%
    }
    \makeatother

    % Exact colors from NB
    \definecolor{incolor}{HTML}{303F9F}
    \definecolor{outcolor}{HTML}{D84315}
    \definecolor{cellborder}{HTML}{CFCFCF}
    \definecolor{cellbackground}{HTML}{F7F7F7}
    
    % prompt
    \makeatletter
    \newcommand{\boxspacing}{\kern\kvtcb@left@rule\kern\kvtcb@boxsep}
    \makeatother
    \newcommand{\prompt}[4]{
        {\ttfamily\llap{{\color{#2}[#3]:\hspace{3pt}#4}}\vspace{-\baselineskip}}
    }
    

    
    % Prevent overflowing lines due to hard-to-break entities
    \sloppy 
    % Setup hyperref package
    \hypersetup{
      breaklinks=true,  % so long urls are correctly broken across lines
      colorlinks=true,
      urlcolor=urlcolor,
      linkcolor=linkcolor,
      citecolor=citecolor,
      }
    % Slightly bigger margins than the latex defaults
    
    \geometry{verbose,tmargin=1in,bmargin=1in,lmargin=1in,rmargin=1in}
    
    

\begin{document}
    
    \maketitle
    
    

    
    \hypertarget{exercise-sheet-10}{%
\section{Exercise sheet 10}\label{exercise-sheet-10}}

\hypertarget{nwo-grant-applications}{%
\subsection{NWO grant applications}\label{nwo-grant-applications}}

\hypertarget{exercise-1}{%
\subsubsection{Exercise 1}\label{exercise-1}}

For this exercise, you will need to load the dataset NWOGrants from the
rethinking package. This dataset reports on the funding outcomes of the
Netherlands Organisation for Scientific Research (NWO) grant
applications over the 2010-2012 period. One study used that dataset to
investigate gender bias in the funding awards:
https://www.pnas.org/content/112/40/12349. You can read there that a
response to this paper has been formulated, as briefly discussed at the
beginning of lecture 9. Note that in general less than half applications
are awarded funding.

Address the same question as the one in the paper by Lee and Ellemers by
using binomial GLM to investigate potential gender bias on awards.
Quantify the contrast between male and female applicants. In a second
step, condition on the discipline as well. What are your conclusions?
Should the NWO take specific measures for gender equity in general and
across disciplines? Is there any excess variance in the data? If yes,
can you elaborate on the nature of possible unaccounted sources?

    \begin{tcolorbox}[breakable, size=fbox, boxrule=1pt, pad at break*=1mm,colback=cellbackground, colframe=cellborder]
\prompt{In}{incolor}{1}{\boxspacing}
\begin{Verbatim}[commandchars=\\\{\}]
\PY{n+nf}{library}\PY{p}{(}\PY{n}{rethinking}\PY{p}{)}
\PY{n+nf}{data}\PY{p}{(}\PY{n}{NWOGrants}\PY{p}{)}
\PY{n}{d} \PY{o}{\PYZlt{}\PYZhy{}} \PY{n}{NWOGrants}
\end{Verbatim}
\end{tcolorbox}

    \begin{Verbatim}[commandchars=\\\{\}]
Lade nötiges Paket: rstan

Lade nötiges Paket: StanHeaders


rstan version 2.26.16 (Stan version 2.26.1)


For execution on a local, multicore CPU with excess RAM we recommend calling
options(mc.cores = parallel::detectCores()).
To avoid recompilation of unchanged Stan programs, we recommend calling
rstan\_options(auto\_write = TRUE)
For within-chain threading using `reduce\_sum()` or `map\_rect()` Stan functions,
change `threads\_per\_chain` option:
rstan\_options(threads\_per\_chain = 1)


Do not specify '-march=native' in 'LOCAL\_CPPFLAGS' or a Makevars file

Lade nötiges Paket: cmdstanr

This is cmdstanr version 0.5.3

- CmdStanR documentation and vignettes: mc-stan.org/cmdstanr

- CmdStan path: D:/Users/Immanuel/Documents/.cmdstan/cmdstan-2.31.0

- CmdStan version: 2.31.0


A newer version of CmdStan is available. See ?install\_cmdstan() to install it.
To disable this check set option or environment variable
CMDSTANR\_NO\_VER\_CHECK=TRUE.

Lade nötiges Paket: parallel

rethinking (Version 2.31)


Attache Paket: 'rethinking'


Das folgende Objekt ist maskiert 'package:rstan':

    stan


Das folgende Objekt ist maskiert 'package:stats':

    rstudent


    \end{Verbatim}

    \begin{tcolorbox}[breakable, size=fbox, boxrule=1pt, pad at break*=1mm,colback=cellbackground, colframe=cellborder]
\prompt{In}{incolor}{2}{\boxspacing}
\begin{Verbatim}[commandchars=\\\{\}]
\PY{n}{d}\PY{o}{\PYZdl{}}\PY{n}{male} \PY{o}{\PYZlt{}\PYZhy{}} \PY{n+nf}{as.integer}\PY{p}{(}\PY{n}{d}\PY{o}{\PYZdl{}}\PY{n}{gender}\PY{p}{)} \PY{o}{\PYZhy{}} \PY{l+m}{1}
\PY{n}{d}\PY{o}{\PYZdl{}}\PY{n}{field} \PY{o}{\PYZlt{}\PYZhy{}} \PY{n+nf}{as.integer}\PY{p}{(}\PY{n}{d}\PY{o}{\PYZdl{}}\PY{n}{discipline}\PY{p}{)}
\PY{n}{d}\PY{o}{\PYZdl{}}\PY{n}{accept} \PY{o}{\PYZlt{}\PYZhy{}} \PY{n}{d}\PY{o}{\PYZdl{}}\PY{n}{awards} \PY{o}{/} \PY{n}{d}\PY{o}{\PYZdl{}}\PY{n}{applications}
\PY{n}{d} \PY{o}{\PYZlt{}\PYZhy{}} \PY{n}{d}\PY{p}{[}\PY{n+nf}{order}\PY{p}{(}\PY{n}{d}\PY{o}{\PYZdl{}}\PY{n}{field}\PY{p}{)}\PY{p}{,} \PY{p}{]}
\PY{n}{d}\PY{o}{\PYZdl{}}\PY{n}{accept.scaled} \PY{o}{\PYZlt{}\PYZhy{}} \PY{n+nf}{scale}\PY{p}{(}\PY{n}{d}\PY{o}{\PYZdl{}}\PY{n}{accept}\PY{p}{)}
\PY{n}{d}
\end{Verbatim}
\end{tcolorbox}

    A data.frame: 18 × 8
\begin{tabular}{r|llllllll}
  & discipline & gender & applications & awards & male & field & accept & accept.scaled\\
  & <fct> & <fct> & <int> & <int> & <dbl> & <int> & <dbl> & <dbl{[},1{]}>\\
\hline
	1 & Chemical sciences   & m &  83 & 22 & 1 & 1 & 0.2650602 &  1.41716126\\
	2 & Chemical sciences   & f &  39 & 10 & 0 & 1 & 0.2564103 &  1.25309407\\
	13 & Earth/life sciences & m & 156 & 38 & 1 & 2 & 0.2435897 &  1.00992305\\
	14 & Earth/life sciences & f & 126 & 18 & 0 & 2 & 0.1428571 & -0.90070635\\
	7 & Humanities          & m & 230 & 33 & 1 & 3 & 0.1434783 & -0.88892539\\
	8 & Humanities          & f & 166 & 32 & 0 & 3 & 0.1927711 &  0.04602831\\
	11 & Interdisciplinary   & m & 105 & 12 & 1 & 4 & 0.1142857 & -1.44263033\\
	12 & Interdisciplinary   & f &  78 & 17 & 0 & 4 & 0.2179487 &  0.52358102\\
	17 & Medical sciences    & m & 245 & 46 & 1 & 5 & 0.1877551 & -0.04911153\\
	18 & Medical sciences    & f & 260 & 29 & 0 & 5 & 0.1115385 & -1.49473840\\
	3 & Physical sciences   & m & 135 & 26 & 1 & 6 & 0.1925926 &  0.04264279\\
	4 & Physical sciences   & f &  39 &  9 & 0 & 6 & 0.2307692 &  0.76675204\\
	5 & Physics             & m &  67 & 18 & 1 & 7 & 0.2686567 &  1.48537683\\
	6 & Physics             & f &   9 &  2 & 0 & 7 & 0.2222222 &  0.60463803\\
	15 & Social sciences     & m & 425 & 65 & 1 & 8 & 0.1529412 & -0.70943907\\
	16 & Social sciences     & f & 409 & 47 & 0 & 8 & 0.1149144 & -1.43070535\\
	9 & Technical sciences  & m & 189 & 30 & 1 & 9 & 0.1587302 & -0.59963748\\
	10 & Technical sciences  & f &  62 & 13 & 0 & 9 & 0.2096774 &  0.36669650\\
\end{tabular}


    
    \begin{tcolorbox}[breakable, size=fbox, boxrule=1pt, pad at break*=1mm,colback=cellbackground, colframe=cellborder]
\prompt{In}{incolor}{3}{\boxspacing}
\begin{Verbatim}[commandchars=\\\{\}]
\PY{n+nf}{par}\PY{p}{(}\PY{n}{mar} \PY{o}{=} \PY{n+nf}{c}\PY{p}{(}\PY{l+m}{4}\PY{p}{,} \PY{l+m}{9}\PY{p}{,} \PY{l+m}{2}\PY{p}{,} \PY{l+m}{2}\PY{p}{)}\PY{p}{)}
\PY{n+nf}{barplot}\PY{p}{(}\PY{n+nf}{rev}\PY{p}{(}\PY{n}{d}\PY{o}{\PYZdl{}}\PY{n}{accept}\PY{p}{)}\PY{p}{,} \PY{n}{names.arg} \PY{o}{=} \PY{n+nf}{rev}\PY{p}{(}\PY{n}{d}\PY{o}{\PYZdl{}}\PY{n}{discipline}\PY{p}{)}\PY{p}{,} \PY{n}{las}\PY{o}{=}\PY{l+m}{2}\PY{p}{,} \PY{n}{xlim} \PY{o}{=} \PY{n+nf}{c}\PY{p}{(}\PY{l+m}{0.10}\PY{p}{,} \PY{l+m}{0.30}\PY{p}{)}\PY{p}{,} \PY{n}{xpd} \PY{o}{=} \PY{k+kc}{FALSE}\PY{p}{,} \PY{n}{horiz} \PY{o}{=} \PY{k+kc}{TRUE}\PY{p}{,} \PY{n}{col} \PY{o}{=} \PY{n+nf}{ifelse}\PY{p}{(}\PY{n}{d}\PY{o}{\PYZdl{}}\PY{n}{male}\PY{p}{,} \PY{l+s}{\PYZdq{}}\PY{l+s}{red\PYZdq{}}\PY{p}{,} \PY{l+s}{\PYZdq{}}\PY{l+s}{blue\PYZdq{}}\PY{p}{)}\PY{p}{)}
\PY{n+nf}{title}\PY{p}{(}\PY{l+s}{\PYZdq{}}\PY{l+s}{Acceptance rate per discipline and gender\PYZdq{}}\PY{p}{)}
\PY{n+nf}{legend}\PY{p}{(}\PY{l+s}{\PYZdq{}}\PY{l+s}{right\PYZdq{}}\PY{p}{,} \PY{n+nf}{c}\PY{p}{(}\PY{l+s}{\PYZdq{}}\PY{l+s}{male\PYZdq{}}\PY{p}{,} \PY{l+s}{\PYZdq{}}\PY{l+s}{female\PYZdq{}}\PY{p}{)}\PY{p}{,} \PY{n}{fill} \PY{o}{=} \PY{n+nf}{c}\PY{p}{(}\PY{l+s}{\PYZdq{}}\PY{l+s}{blue\PYZdq{}}\PY{p}{,} \PY{l+s}{\PYZdq{}}\PY{l+s}{red\PYZdq{}}\PY{p}{)}\PY{p}{)}
\end{Verbatim}
\end{tcolorbox}

    \begin{center}
    \adjustimage{max size={0.9\linewidth}{0.9\paperheight}}{exercise_sheet_10_Immanuel_Albrecht_files/exercise_sheet_10_Immanuel_Albrecht_3_0.png}
    \end{center}
    { \hspace*{\fill} \\}
    
    So male applicants are more likely to get accepted in humanities,
interdiciplinary, physical and technical sciences. Female applicants are
more likely to get accepted in chemical, earth/life, medical and social
sciences and in physics.

    \begin{tcolorbox}[breakable, size=fbox, boxrule=1pt, pad at break*=1mm,colback=cellbackground, colframe=cellborder]
\prompt{In}{incolor}{4}{\boxspacing}
\begin{Verbatim}[commandchars=\\\{\}]
\PY{n}{e} \PY{o}{\PYZlt{}\PYZhy{}} \PY{n+nf}{data.frame}\PY{p}{(}\PY{n}{discipline} \PY{o}{=} \PY{n+nf}{character}\PY{p}{(}\PY{p}{)}\PY{p}{,} \PY{n}{applications} \PY{o}{=} \PY{n+nf}{integer}\PY{p}{(}\PY{p}{)}\PY{p}{,} \PY{n}{awards} \PY{o}{=} \PY{n+nf}{integer}\PY{p}{(}\PY{p}{)}\PY{p}{,} \PY{n}{field} \PY{o}{=} \PY{n+nf}{integer}\PY{p}{(}\PY{p}{)}\PY{p}{,} \PY{n}{accept} \PY{o}{=} \PY{n+nf}{double}\PY{p}{(}\PY{p}{)}\PY{p}{)}
\PY{n+nf}{for }\PY{p}{(}\PY{n}{field} \PY{n}{in} \PY{l+m}{1}\PY{o}{:}\PY{n+nf}{max}\PY{p}{(}\PY{n}{d}\PY{o}{\PYZdl{}}\PY{n}{field}\PY{p}{)}\PY{p}{)} \PY{p}{\PYZob{}}
	\PY{n}{dat} \PY{o}{\PYZlt{}\PYZhy{}} \PY{n}{d}\PY{p}{[}\PY{n}{d}\PY{o}{\PYZdl{}}\PY{n}{field} \PY{o}{==} \PY{n}{field}\PY{p}{,} \PY{p}{]}
	\PY{n}{discipline} \PY{o}{\PYZlt{}\PYZhy{}} \PY{n+nf}{as.character}\PY{p}{(}\PY{n}{dat}\PY{o}{\PYZdl{}}\PY{n}{discipline}\PY{p}{[}\PY{l+m}{1}\PY{p}{]}\PY{p}{)}
	\PY{n}{applications} \PY{o}{\PYZlt{}\PYZhy{}} \PY{n+nf}{sum}\PY{p}{(}\PY{n}{dat}\PY{o}{\PYZdl{}}\PY{n}{applications}\PY{p}{)}
	\PY{n}{awards} \PY{o}{\PYZlt{}\PYZhy{}} \PY{n+nf}{sum}\PY{p}{(}\PY{n}{dat}\PY{o}{\PYZdl{}}\PY{n}{awards}\PY{p}{)}
	\PY{n}{accept} \PY{o}{\PYZlt{}\PYZhy{}} \PY{n}{awards} \PY{o}{/} \PY{n}{applications}
	\PY{n}{e}\PY{p}{[}\PY{n+nf}{nrow}\PY{p}{(}\PY{n}{e}\PY{p}{)} \PY{o}{+} \PY{l+m}{1}\PY{p}{,} \PY{p}{]} \PY{o}{\PYZlt{}\PYZhy{}} \PY{n+nf}{list}\PY{p}{(}\PY{n}{discipline}\PY{p}{,} \PY{n}{applications}\PY{p}{,} \PY{n}{awards}\PY{p}{,} \PY{n}{field}\PY{p}{,} \PY{n}{accept}\PY{p}{)}
\PY{p}{\PYZcb{}}
\PY{n}{e}
\end{Verbatim}
\end{tcolorbox}

    A data.frame: 9 × 5
\begin{tabular}{r|lllll}
  & discipline & applications & awards & field & accept\\
  & <chr> & <int> & <int> & <int> & <dbl>\\
\hline
	1 & Chemical sciences   & 122 &  32 & 1 & 0.2622951\\
	2 & Earth/life sciences & 282 &  56 & 2 & 0.1985816\\
	3 & Humanities          & 396 &  65 & 3 & 0.1641414\\
	4 & Interdisciplinary   & 183 &  29 & 4 & 0.1584699\\
	5 & Medical sciences    & 505 &  75 & 5 & 0.1485149\\
	6 & Physical sciences   & 174 &  35 & 6 & 0.2011494\\
	7 & Physics             &  76 &  20 & 7 & 0.2631579\\
	8 & Social sciences     & 834 & 112 & 8 & 0.1342926\\
	9 & Technical sciences  & 251 &  43 & 9 & 0.1713147\\
\end{tabular}


    
    \begin{tcolorbox}[breakable, size=fbox, boxrule=1pt, pad at break*=1mm,colback=cellbackground, colframe=cellborder]
\prompt{In}{incolor}{5}{\boxspacing}
\begin{Verbatim}[commandchars=\\\{\}]
\PY{n+nf}{par}\PY{p}{(}\PY{n}{mar} \PY{o}{=} \PY{n+nf}{c}\PY{p}{(}\PY{l+m}{4}\PY{p}{,} \PY{l+m}{9}\PY{p}{,} \PY{l+m}{2}\PY{p}{,} \PY{l+m}{2}\PY{p}{)}\PY{p}{)}
\PY{n+nf}{barplot}\PY{p}{(}\PY{n}{e}\PY{o}{\PYZdl{}}\PY{n}{accept}\PY{p}{,} \PY{n}{names.arg} \PY{o}{=} \PY{n}{e}\PY{o}{\PYZdl{}}\PY{n}{discipline}\PY{p}{,} \PY{n}{las}\PY{o}{=}\PY{l+m}{2}\PY{p}{,} \PY{n}{xlim} \PY{o}{=} \PY{n+nf}{c}\PY{p}{(}\PY{l+m}{0.10}\PY{p}{,} \PY{l+m}{0.30}\PY{p}{)}\PY{p}{,} \PY{n}{xpd} \PY{o}{=} \PY{k+kc}{FALSE}\PY{p}{,} \PY{n}{horiz} \PY{o}{=} \PY{k+kc}{TRUE}\PY{p}{)}
\PY{n+nf}{title}\PY{p}{(}\PY{l+s}{\PYZdq{}}\PY{l+s}{Acceptance rate per discipline\PYZdq{}}\PY{p}{)}
\end{Verbatim}
\end{tcolorbox}

    \begin{center}
    \adjustimage{max size={0.9\linewidth}{0.9\paperheight}}{exercise_sheet_10_Immanuel_Albrecht_files/exercise_sheet_10_Immanuel_Albrecht_6_0.png}
    \end{center}
    { \hspace*{\fill} \\}
    
    \begin{tcolorbox}[breakable, size=fbox, boxrule=1pt, pad at break*=1mm,colback=cellbackground, colframe=cellborder]
\prompt{In}{incolor}{6}{\boxspacing}
\begin{Verbatim}[commandchars=\\\{\}]
\PY{n}{male\PYZus{}applications} \PY{o}{\PYZlt{}\PYZhy{}} \PY{n+nf}{sum}\PY{p}{(}\PY{n}{d}\PY{p}{[}\PY{n}{d}\PY{o}{\PYZdl{}}\PY{n}{male} \PY{o}{==} \PY{l+m}{1}\PY{p}{,} \PY{p}{]}\PY{o}{\PYZdl{}}\PY{n}{applications}\PY{p}{)}
\PY{n}{female\PYZus{}applications} \PY{o}{\PYZlt{}\PYZhy{}} \PY{n+nf}{sum}\PY{p}{(}\PY{n}{d}\PY{p}{[}\PY{n}{d}\PY{o}{\PYZdl{}}\PY{n}{male} \PY{o}{==} \PY{l+m}{0}\PY{p}{,} \PY{p}{]}\PY{o}{\PYZdl{}}\PY{n}{applications}\PY{p}{)}
\PY{n}{male\PYZus{}awards} \PY{o}{\PYZlt{}\PYZhy{}} \PY{n+nf}{sum}\PY{p}{(}\PY{n}{d}\PY{p}{[}\PY{n}{d}\PY{o}{\PYZdl{}}\PY{n}{male} \PY{o}{==} \PY{l+m}{1}\PY{p}{,} \PY{p}{]}\PY{o}{\PYZdl{}}\PY{n}{awards}\PY{p}{)}
\PY{n}{female\PYZus{}awards} \PY{o}{\PYZlt{}\PYZhy{}} \PY{n+nf}{sum}\PY{p}{(}\PY{n}{d}\PY{p}{[}\PY{n}{d}\PY{o}{\PYZdl{}}\PY{n}{male} \PY{o}{==} \PY{l+m}{0}\PY{p}{,} \PY{p}{]}\PY{o}{\PYZdl{}}\PY{n}{awards}\PY{p}{)}
\PY{n}{male\PYZus{}accept} \PY{o}{\PYZlt{}\PYZhy{}} \PY{n}{male\PYZus{}awards} \PY{o}{/} \PY{n}{male\PYZus{}applications}
\PY{n}{female\PYZus{}accept} \PY{o}{\PYZlt{}\PYZhy{}} \PY{n}{female\PYZus{}awards} \PY{o}{/} \PY{n}{female\PYZus{}applications}

\PY{n+nf}{barplot}\PY{p}{(}\PY{n+nf}{c}\PY{p}{(}\PY{n}{male\PYZus{}accept}\PY{p}{,} \PY{n}{female\PYZus{}accept}\PY{p}{)}\PY{p}{,} \PY{n}{names.arg} \PY{o}{=} \PY{n+nf}{c}\PY{p}{(}\PY{l+s}{\PYZdq{}}\PY{l+s}{male\PYZdq{}}\PY{p}{,} \PY{l+s}{\PYZdq{}}\PY{l+s}{female\PYZdq{}}\PY{p}{)}\PY{p}{,} \PY{n}{ylim} \PY{o}{=} \PY{n+nf}{c}\PY{p}{(}\PY{l+m}{0.14}\PY{p}{,} \PY{l+m}{0.18}\PY{p}{)}\PY{p}{,} \PY{n}{xpd} \PY{o}{=} \PY{k+kc}{FALSE}\PY{p}{,} \PY{n}{col} \PY{o}{=} \PY{n+nf}{c}\PY{p}{(}\PY{l+s}{\PYZdq{}}\PY{l+s}{blue\PYZdq{}}\PY{p}{,} \PY{l+s}{\PYZdq{}}\PY{l+s}{red\PYZdq{}}\PY{p}{)}\PY{p}{)}
\PY{n+nf}{title}\PY{p}{(}\PY{l+s}{\PYZdq{}}\PY{l+s}{Acceptance rate per gender\PYZdq{}}\PY{p}{)}

\PY{n+nf}{mean}\PY{p}{(}\PY{n}{male\PYZus{}accept}\PY{p}{)} \PY{o}{\PYZhy{}} \PY{n+nf}{mean}\PY{p}{(}\PY{n}{female\PYZus{}accept}\PY{p}{)}
\end{Verbatim}
\end{tcolorbox}

    0.0283801315911408

    
    \begin{center}
    \adjustimage{max size={0.9\linewidth}{0.9\paperheight}}{exercise_sheet_10_Immanuel_Albrecht_files/exercise_sheet_10_Immanuel_Albrecht_7_1.png}
    \end{center}
    { \hspace*{\fill} \\}
    
    So in the data there is a difference between the male and female
acceptance rate in the data. But is it statistically significant? We
construct different models and compare them.

    \begin{tcolorbox}[breakable, size=fbox, boxrule=1pt, pad at break*=1mm,colback=cellbackground, colframe=cellborder]
\prompt{In}{incolor}{7}{\boxspacing}
\begin{Verbatim}[commandchars=\\\{\}]
\PY{c+c1}{\PYZsh{} A single mean over all the data}
\PY{n}{model1} \PY{o}{\PYZlt{}\PYZhy{}} \PY{n+nf}{map}\PY{p}{(}
	\PY{n+nf}{alist}\PY{p}{(}
		\PY{n}{awards} \PY{o}{\PYZti{}} \PY{n+nf}{dbinom}\PY{p}{(}\PY{n}{applications}\PY{p}{,} \PY{n}{p}\PY{p}{)}\PY{p}{,}
		\PY{n+nf}{logit}\PY{p}{(}\PY{n}{p}\PY{p}{)} \PY{o}{\PYZlt{}\PYZhy{}} \PY{n}{a}\PY{p}{,}
		\PY{n}{a} \PY{o}{\PYZti{}} \PY{n+nf}{dnorm}\PY{p}{(}\PY{l+m}{0}\PY{p}{,} \PY{l+m}{10}\PY{p}{)}
	\PY{p}{)}\PY{p}{,}
	\PY{n}{data} \PY{o}{=} \PY{n}{d}
\PY{p}{)}
\PY{n}{model1\PYZus{}parameter\PYZus{}amount} \PY{o}{\PYZlt{}\PYZhy{}} \PY{l+m}{1} \PY{c+c1}{\PYZsh{} 1}
\end{Verbatim}
\end{tcolorbox}

    \begin{tcolorbox}[breakable, size=fbox, boxrule=1pt, pad at break*=1mm,colback=cellbackground, colframe=cellborder]
\prompt{In}{incolor}{8}{\boxspacing}
\begin{Verbatim}[commandchars=\\\{\}]
\PY{c+c1}{\PYZsh{} One mean for male and one mean for female}
\PY{n}{model\PYZus{}male} \PY{o}{\PYZlt{}\PYZhy{}} \PY{n+nf}{map}\PY{p}{(}
	\PY{n+nf}{alist}\PY{p}{(}
		\PY{n}{awards} \PY{o}{\PYZti{}} \PY{n+nf}{dbinom}\PY{p}{(}\PY{n}{applications}\PY{p}{,} \PY{n}{p}\PY{p}{)}\PY{p}{,}
		\PY{n+nf}{logit}\PY{p}{(}\PY{n}{p}\PY{p}{)} \PY{o}{\PYZlt{}\PYZhy{}} \PY{n}{a} \PY{o}{+} \PY{n}{a\PYZus{}male} \PY{o}{*} \PY{n}{male}\PY{p}{,}
		\PY{n}{a} \PY{o}{\PYZti{}} \PY{n+nf}{dnorm}\PY{p}{(}\PY{l+m}{0}\PY{p}{,} \PY{l+m}{10}\PY{p}{)}\PY{p}{,}
		\PY{n}{a\PYZus{}male} \PY{o}{\PYZti{}} \PY{n+nf}{dnorm}\PY{p}{(}\PY{l+m}{0}\PY{p}{,} \PY{l+m}{10}\PY{p}{)}
	\PY{p}{)}\PY{p}{,}
	\PY{n}{data} \PY{o}{=} \PY{n}{d}
\PY{p}{)}
\PY{n}{model\PYZus{}male\PYZus{}parameter\PYZus{}amount} \PY{o}{\PYZlt{}\PYZhy{}} \PY{l+m}{1} \PY{o}{+} \PY{l+m}{1} \PY{c+c1}{\PYZsh{} 2}
\end{Verbatim}
\end{tcolorbox}

    \begin{tcolorbox}[breakable, size=fbox, boxrule=1pt, pad at break*=1mm,colback=cellbackground, colframe=cellborder]
\prompt{In}{incolor}{9}{\boxspacing}
\begin{Verbatim}[commandchars=\\\{\}]
\PY{c+c1}{\PYZsh{} A model where we dont distinguish between male and female, but the department is relevant}
\PY{n}{model\PYZus{}field} \PY{o}{\PYZlt{}\PYZhy{}} \PY{n+nf}{map}\PY{p}{(}
	\PY{n+nf}{alist}\PY{p}{(}
		\PY{n}{awards} \PY{o}{\PYZti{}} \PY{n+nf}{dbinom}\PY{p}{(}\PY{n}{applications}\PY{p}{,} \PY{n}{p}\PY{p}{)}\PY{p}{,}
		\PY{n+nf}{logit}\PY{p}{(}\PY{n}{p}\PY{p}{)} \PY{o}{\PYZlt{}\PYZhy{}} \PY{n}{a}\PY{p}{[}\PY{n}{field}\PY{p}{]}\PY{p}{,}
		\PY{n}{a}\PY{p}{[}\PY{n}{field}\PY{p}{]} \PY{o}{\PYZti{}} \PY{n+nf}{dnorm}\PY{p}{(}\PY{l+m}{0}\PY{p}{,} \PY{l+m}{10}\PY{p}{)}
	\PY{p}{)}\PY{p}{,}
	\PY{n}{data} \PY{o}{=} \PY{n}{d}
\PY{p}{)}
\PY{n}{model\PYZus{}field\PYZus{}parameter\PYZus{}amount} \PY{o}{\PYZlt{}\PYZhy{}} \PY{n+nf}{nrow}\PY{p}{(}\PY{n}{e}\PY{p}{)} \PY{c+c1}{\PYZsh{} 9}
\end{Verbatim}
\end{tcolorbox}

    \begin{tcolorbox}[breakable, size=fbox, boxrule=1pt, pad at break*=1mm,colback=cellbackground, colframe=cellborder]
\prompt{In}{incolor}{10}{\boxspacing}
\begin{Verbatim}[commandchars=\\\{\}]
\PY{c+c1}{\PYZsh{} Each field has its own mean but the change if a participant is male is global}
\PY{n}{model\PYZus{}field\PYZus{}male} \PY{o}{\PYZlt{}\PYZhy{}} \PY{n+nf}{map}\PY{p}{(}
	\PY{n+nf}{alist}\PY{p}{(}
		\PY{n}{awards} \PY{o}{\PYZti{}} \PY{n+nf}{dbinom}\PY{p}{(}\PY{n}{applications}\PY{p}{,} \PY{n}{p}\PY{p}{)}\PY{p}{,}
		\PY{n+nf}{logit}\PY{p}{(}\PY{n}{p}\PY{p}{)} \PY{o}{\PYZlt{}\PYZhy{}} \PY{n}{a}\PY{p}{[}\PY{n}{field}\PY{p}{]} \PY{o}{+} \PY{n}{a\PYZus{}male} \PY{o}{*} \PY{n}{male}\PY{p}{,}
		\PY{n}{a}\PY{p}{[}\PY{n}{field}\PY{p}{]} \PY{o}{\PYZti{}} \PY{n+nf}{dnorm}\PY{p}{(}\PY{l+m}{0}\PY{p}{,} \PY{l+m}{10}\PY{p}{)}\PY{p}{,}
		\PY{n}{a\PYZus{}male} \PY{o}{\PYZti{}} \PY{n+nf}{dnorm}\PY{p}{(}\PY{l+m}{0}\PY{p}{,} \PY{l+m}{10}\PY{p}{)}
	\PY{p}{)}\PY{p}{,}
	\PY{n}{data} \PY{o}{=} \PY{n}{d}
\PY{p}{)}
\PY{n}{model\PYZus{}field\PYZus{}male\PYZus{}parameter\PYZus{}amount} \PY{o}{\PYZlt{}\PYZhy{}} \PY{n+nf}{nrow}\PY{p}{(}\PY{n}{e}\PY{p}{)} \PY{o}{+} \PY{l+m}{1} \PY{c+c1}{\PYZsh{} 10}
\end{Verbatim}
\end{tcolorbox}

    \begin{tcolorbox}[breakable, size=fbox, boxrule=1pt, pad at break*=1mm,colback=cellbackground, colframe=cellborder]
\prompt{In}{incolor}{11}{\boxspacing}
\begin{Verbatim}[commandchars=\\\{\}]
\PY{c+c1}{\PYZsh{} We have the same mean for all fields, but the influence if the participant is male is unique for each field}
\PY{n}{model\PYZus{}male\PYZus{}field} \PY{o}{\PYZlt{}\PYZhy{}} \PY{n+nf}{map}\PY{p}{(}
	\PY{n+nf}{alist}\PY{p}{(}
		\PY{n}{awards} \PY{o}{\PYZti{}} \PY{n+nf}{dbinom}\PY{p}{(}\PY{n}{applications}\PY{p}{,} \PY{n}{p}\PY{p}{)}\PY{p}{,}
		\PY{n+nf}{logit}\PY{p}{(}\PY{n}{p}\PY{p}{)} \PY{o}{\PYZlt{}\PYZhy{}} \PY{n}{a} \PY{o}{+} \PY{n}{a\PYZus{}male}\PY{p}{[}\PY{n}{field}\PY{p}{]} \PY{o}{*} \PY{n}{male}\PY{p}{,}
		\PY{n}{a} \PY{o}{\PYZti{}} \PY{n+nf}{dnorm}\PY{p}{(}\PY{l+m}{0}\PY{p}{,} \PY{l+m}{10}\PY{p}{)}\PY{p}{,}
		\PY{n}{a\PYZus{}male}\PY{p}{[}\PY{n}{field}\PY{p}{]} \PY{o}{\PYZti{}} \PY{n+nf}{dnorm}\PY{p}{(}\PY{l+m}{0}\PY{p}{,} \PY{l+m}{10}\PY{p}{)}
	\PY{p}{)}\PY{p}{,}
	\PY{n}{data} \PY{o}{=} \PY{n}{d}
\PY{p}{)}
\PY{n}{model\PYZus{}male\PYZus{}field\PYZus{}parameter\PYZus{}amount} \PY{o}{\PYZlt{}\PYZhy{}} \PY{l+m}{1} \PY{o}{+} \PY{n+nf}{nrow}\PY{p}{(}\PY{n}{e}\PY{p}{)} \PY{c+c1}{\PYZsh{} 10}
\end{Verbatim}
\end{tcolorbox}

    \begin{tcolorbox}[breakable, size=fbox, boxrule=1pt, pad at break*=1mm,colback=cellbackground, colframe=cellborder]
\prompt{In}{incolor}{12}{\boxspacing}
\begin{Verbatim}[commandchars=\\\{\}]
\PY{c+c1}{\PYZsh{} Both the mean and the offset for a male participant is different for each department}
\PY{n}{model\PYZus{}field\PYZus{}male\PYZus{}field} \PY{o}{\PYZlt{}\PYZhy{}} \PY{n+nf}{map}\PY{p}{(}
	\PY{n+nf}{alist}\PY{p}{(}
		\PY{n}{awards} \PY{o}{\PYZti{}} \PY{n+nf}{dbinom}\PY{p}{(}\PY{n}{applications}\PY{p}{,} \PY{n}{p}\PY{p}{)}\PY{p}{,}
		\PY{n+nf}{logit}\PY{p}{(}\PY{n}{p}\PY{p}{)} \PY{o}{\PYZlt{}\PYZhy{}} \PY{n}{a}\PY{p}{[}\PY{n}{field}\PY{p}{]} \PY{o}{+} \PY{n}{a\PYZus{}male}\PY{p}{[}\PY{n}{field}\PY{p}{]} \PY{o}{*} \PY{n}{male}\PY{p}{,}
		\PY{n}{a}\PY{p}{[}\PY{n}{field}\PY{p}{]} \PY{o}{\PYZti{}} \PY{n+nf}{dnorm}\PY{p}{(}\PY{l+m}{0}\PY{p}{,} \PY{l+m}{10}\PY{p}{)}\PY{p}{,}
		\PY{n}{a\PYZus{}male}\PY{p}{[}\PY{n}{field}\PY{p}{]} \PY{o}{\PYZti{}} \PY{n+nf}{dnorm}\PY{p}{(}\PY{l+m}{0}\PY{p}{,} \PY{l+m}{10}\PY{p}{)}
	\PY{p}{)}\PY{p}{,}
	\PY{n}{data} \PY{o}{=} \PY{n}{d}
\PY{p}{)}
\PY{n}{model6\PYZus{}field\PYZus{}male\PYZus{}field\PYZus{}parameter\PYZus{}amount} \PY{o}{\PYZlt{}\PYZhy{}} \PY{n+nf}{nrow}\PY{p}{(}\PY{n}{e}\PY{p}{)} \PY{o}{+} \PY{n+nf}{nrow}\PY{p}{(}\PY{n}{e}\PY{p}{)} \PY{c+c1}{\PYZsh{} 18}
\end{Verbatim}
\end{tcolorbox}

    \begin{tcolorbox}[breakable, size=fbox, boxrule=1pt, pad at break*=1mm,colback=cellbackground, colframe=cellborder]
\prompt{In}{incolor}{13}{\boxspacing}
\begin{Verbatim}[commandchars=\\\{\}]
\PY{n+nf}{compare}\PY{p}{(}\PY{n}{model1}\PY{p}{,} \PY{n}{model\PYZus{}male}\PY{p}{,} \PY{n}{model\PYZus{}field}\PY{p}{,} \PY{n}{model\PYZus{}field\PYZus{}male}\PY{p}{,} \PY{n}{model\PYZus{}male\PYZus{}field}\PY{p}{,} \PY{n}{model\PYZus{}field\PYZus{}male\PYZus{}field}\PY{p}{)}
\end{Verbatim}
\end{tcolorbox}

    A compareIC: 6 × 6
\begin{tabular}{r|llllll}
  & WAIC & SE & dWAIC & dSE & pWAIC & weight\\
  & <dbl> & <dbl> & <dbl> & <dbl> & <dbl> & <dbl>\\
\hline
	model\_field\_male\_field & 114.1189 &  3.323187 &  0.000000 &        NA &  8.916445 & 0.9780031254\\
	model\_male\_field & 121.8910 &  7.573247 &  7.772083 &  5.880442 &  7.050472 & 0.0200749240\\
	model\_field\_male & 128.4570 &  8.013722 & 14.338031 &  6.176659 & 12.589328 & 0.0007531413\\
	model\_field & 129.1645 &  9.418417 & 15.045598 &  7.314806 & 12.477572 & 0.0005287253\\
	model\_male & 130.1521 &  8.867268 & 16.033179 &  7.403613 &  4.998377 & 0.0003226857\\
	model1 & 130.1852 & 12.900592 & 16.066222 & 11.779835 &  2.702629 & 0.0003173983\\
\end{tabular}


    
    So we see that the model that is able to distinguish between male and
female and also between departments is able to fit the data the best,
obviously.

    Now we quantify the contrast while disregarding the field the
participants come from. These are the two models \texttt{model\_male}
and \texttt{model1}. Because the latter is gender independent, we only
have to do this discussion one for the first model.

    \begin{tcolorbox}[breakable, size=fbox, boxrule=1pt, pad at break*=1mm,colback=cellbackground, colframe=cellborder]
\prompt{In}{incolor}{14}{\boxspacing}
\begin{Verbatim}[commandchars=\\\{\}]
\PY{c+c1}{\PYZsh{} The relative scale, which in most cases can be disregarded.}
\PY{n+nf}{summary}\PY{p}{(}\PY{n}{model1}\PY{p}{)}
\PY{n+nf}{summary}\PY{p}{(}\PY{n}{model\PYZus{}male}\PY{p}{)}
\end{Verbatim}
\end{tcolorbox}

    A precis: 1 × 4
\begin{tabular}{r|llll}
  & mean & sd & 5.5\% & 94.5\%\\
  & <dbl> & <dbl> & <dbl> & <dbl>\\
\hline
	a & -1.61835 & 0.05065213 & -1.699302 & -1.537398\\
\end{tabular}


    
    A precis: 2 × 4
\begin{tabular}{r|llll}
  & mean & sd & 5.5\% & 94.5\%\\
  & <dbl> & <dbl> & <dbl> & <dbl>\\
\hline
	a & -1.7424160 & 0.08146995 & -1.87262075 & -1.6122113\\
	a\_male &  0.2081389 & 0.10405967 &  0.04183145 &  0.3744463\\
\end{tabular}


    
    \begin{tcolorbox}[breakable, size=fbox, boxrule=1pt, pad at break*=1mm,colback=cellbackground, colframe=cellborder]
\prompt{In}{incolor}{15}{\boxspacing}
\begin{Verbatim}[commandchars=\\\{\}]
\PY{c+c1}{\PYZsh{} Better to compare the models with an information criterion}
\PY{n+nf}{compare}\PY{p}{(}\PY{n}{model1}\PY{p}{,} \PY{n}{model\PYZus{}male}\PY{p}{)}
\end{Verbatim}
\end{tcolorbox}

    A compareIC: 2 × 6
\begin{tabular}{r|llllll}
  & WAIC & SE & dWAIC & dSE & pWAIC & weight\\
  & <dbl> & <dbl> & <dbl> & <dbl> & <dbl> & <dbl>\\
\hline
	model\_male & 129.9961 &  8.736106 & 0.0000000 &       NA & 4.900122 & 0.5569528\\
	model1 & 130.4537 & 13.016595 & 0.4576087 & 6.713701 & 2.867156 & 0.4430472\\
\end{tabular}


    
    We see that the inclusion of the male parameter makes a difference,
although its not that significant. Next we look at the contrast between
the number of male and female awardees in the second model.

    \begin{tcolorbox}[breakable, size=fbox, boxrule=1pt, pad at break*=1mm,colback=cellbackground, colframe=cellborder]
\prompt{In}{incolor}{16}{\boxspacing}
\begin{Verbatim}[commandchars=\\\{\}]
\PY{n}{post} \PY{o}{\PYZlt{}\PYZhy{}} \PY{n+nf}{extract.samples}\PY{p}{(}\PY{n}{model\PYZus{}male}\PY{p}{)}
\PY{n}{p.award.male} \PY{o}{\PYZlt{}\PYZhy{}} \PY{n+nf}{logistic}\PY{p}{(}\PY{n}{post}\PY{o}{\PYZdl{}}\PY{n}{a} \PY{o}{+} \PY{n}{post}\PY{o}{\PYZdl{}}\PY{n}{a\PYZus{}male}\PY{p}{)}
\PY{n}{p.award.female} \PY{o}{\PYZlt{}\PYZhy{}} \PY{n+nf}{logistic}\PY{p}{(}\PY{n}{post}\PY{o}{\PYZdl{}}\PY{n}{a}\PY{p}{)}
\PY{n}{diff.award} \PY{o}{\PYZlt{}\PYZhy{}} \PY{n}{p.award.male} \PY{o}{\PYZhy{}} \PY{n}{p.award.female}
\PY{n+nf}{quantile}\PY{p}{(}\PY{n}{diff.award}\PY{p}{,} \PY{n+nf}{c}\PY{p}{(}\PY{l+m}{0.025}\PY{p}{,} \PY{l+m}{0.5}\PY{p}{,} \PY{l+m}{0.975}\PY{p}{)}\PY{p}{)}
\end{Verbatim}
\end{tcolorbox}

    \begin{description*}
\item[2.5\textbackslash{}\%] 0.000513055530130447
\item[50\textbackslash{}\%] 0.0286093141054553
\item[97.5\textbackslash{}\%] 0.0556196194455272
\end{description*}


    
    We see that as in the data, the model generated a gap between male and
female of 0.028, which means that males are 3 percent more likely to be
awarded than females. Next we also include the field of study. We
consider the model \texttt{model\_field\_male\_field}.

    \begin{tcolorbox}[breakable, size=fbox, boxrule=1pt, pad at break*=1mm,colback=cellbackground, colframe=cellborder]
\prompt{In}{incolor}{17}{\boxspacing}
\begin{Verbatim}[commandchars=\\\{\}]
\PY{n}{post} \PY{o}{\PYZlt{}\PYZhy{}} \PY{n+nf}{extract.samples}\PY{p}{(}\PY{n}{model\PYZus{}field\PYZus{}male\PYZus{}field}\PY{p}{)}
\PY{n}{p.award.male} \PY{o}{\PYZlt{}\PYZhy{}} \PY{n+nf}{logistic}\PY{p}{(}\PY{n}{post}\PY{o}{\PYZdl{}}\PY{n}{a} \PY{o}{+} \PY{n}{post}\PY{o}{\PYZdl{}}\PY{n}{a\PYZus{}male}\PY{p}{)}
\PY{n}{p.award.female} \PY{o}{\PYZlt{}\PYZhy{}} \PY{n+nf}{logistic}\PY{p}{(}\PY{n}{post}\PY{o}{\PYZdl{}}\PY{n}{a}\PY{p}{)}
\PY{n}{diff.award} \PY{o}{\PYZlt{}\PYZhy{}} \PY{n}{p.award.male} \PY{o}{\PYZhy{}} \PY{n}{p.award.female}

\PY{n}{means} \PY{o}{\PYZlt{}\PYZhy{}} \PY{n+nf}{c}\PY{p}{(}\PY{p}{)}
\PY{n}{names} \PY{o}{\PYZlt{}\PYZhy{}} \PY{n+nf}{c}\PY{p}{(}\PY{p}{)}
\PY{n+nf}{for }\PY{p}{(}\PY{n}{index} \PY{n}{in} \PY{l+m}{1}\PY{o}{:}\PY{n+nf}{ncol}\PY{p}{(}\PY{n}{diff.award}\PY{p}{)}\PY{p}{)} \PY{p}{\PYZob{}}
	\PY{n}{means} \PY{o}{\PYZlt{}\PYZhy{}} \PY{n+nf}{append}\PY{p}{(}\PY{n}{means}\PY{p}{,} \PY{n+nf}{mean}\PY{p}{(}\PY{n}{diff.award}\PY{p}{[}\PY{p}{,} \PY{n}{index}\PY{p}{]}\PY{p}{)}\PY{p}{)}
	\PY{n}{names} \PY{o}{\PYZlt{}\PYZhy{}} \PY{n+nf}{append}\PY{p}{(}\PY{n}{names}\PY{p}{,} \PY{n}{e}\PY{p}{[}\PY{n}{e}\PY{o}{\PYZdl{}}\PY{n}{field} \PY{o}{==} \PY{n}{index}\PY{p}{,} \PY{p}{]}\PY{o}{\PYZdl{}}\PY{n}{discipline}\PY{p}{)}
\PY{p}{\PYZcb{}}

\PY{n+nf}{par}\PY{p}{(}\PY{n}{mar} \PY{o}{=} \PY{n+nf}{c}\PY{p}{(}\PY{l+m}{4}\PY{p}{,} \PY{l+m}{9}\PY{p}{,} \PY{l+m}{2}\PY{p}{,} \PY{l+m}{2}\PY{p}{)}\PY{p}{)}
\PY{n}{bp} \PY{o}{\PYZlt{}\PYZhy{}} \PY{n+nf}{barplot}\PY{p}{(}\PY{n}{means}\PY{p}{,} \PY{n}{names.arg} \PY{o}{=} \PY{n}{names}\PY{p}{,} \PY{n}{las}\PY{o}{=}\PY{l+m}{2}\PY{p}{,} \PY{n}{xpd} \PY{o}{=} \PY{k+kc}{FALSE}\PY{p}{,} \PY{n}{horiz} \PY{o}{=} \PY{k+kc}{TRUE}\PY{p}{,} \PY{n}{xlim} \PY{o}{=} \PY{n+nf}{c}\PY{p}{(}\PY{l+m}{\PYZhy{}0.11}\PY{p}{,} \PY{l+m}{0.11}\PY{p}{)}\PY{p}{)}
\PY{n+nf}{title}\PY{p}{(}\PY{l+s}{\PYZdq{}}\PY{l+s}{Award probability gender difference per discipline\PYZdq{}}\PY{p}{)}
\PY{n+nf}{text}\PY{p}{(}\PY{n}{bp}\PY{p}{,} \PY{l+m}{0}\PY{p}{,} \PY{n}{means}\PY{p}{)}
\end{Verbatim}
\end{tcolorbox}

    \begin{center}
    \adjustimage{max size={0.9\linewidth}{0.9\paperheight}}{exercise_sheet_10_Immanuel_Albrecht_files/exercise_sheet_10_Immanuel_Albrecht_23_0.png}
    \end{center}
    { \hspace*{\fill} \\}
    

    % Add a bibliography block to the postdoc
    
    
    
\end{document}
